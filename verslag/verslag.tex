\documentclass[12pt]{article}

\usepackage{graphicx}
\usepackage{fullpage}

\title{TI2200 Practical report\\
Assignment 2 (AIS-E)}
\author{Michiel Haisma 1512285\\
Joost Meulenbeld \\
Boy Bos\\
Leon Loopik\\
TU Delft\\}
\date{\today}


\begin{document}

\maketitle

\section*{Assignment 1}
Hallo bladiebla

\begin{verbatim}
hallo dit
is 
	code
\end{verbatim}


\section*{Assignment 2}

\section*{Assignment 3}

\section*{Assignment 4}

\section*{Assignment 5}

\section*{Assignment 6}

\section*{Assignment 7}
Any kind of artifact, especially documents, can be extremely usefull to have when working on a software system. The documents should provide \emph{all} information and in addition to the program code, should offer the developer a complete view of the system. This system was delivered with only one document: the installation guide. However very usefull for the users who wish to use the system, it offers little information for developers who want to look into the system.

\begin{itemize}
\item System overview\\
A simple document describing how the system is built on a high level. This document should specify the modules that are used to build the system. Also this should show for instance: the user and OpenMaps integration. The goal of this document is to show how different parts are connected (not how they communicate per se) so the developer knows how system design and domain concepts are seperated and how the system interacts with other systems.

\item Class diagram\\
A class diagram provides an excellent insight in how the system is actually built. It should be very specific, every attribute and method should be shown, as well as relations between classes. This is very usefull to see what the relations between classes are and how they communicate. The class diagram can largely be derived from the code. This makes the class diagram almost a visual implementation of the system.

\item Sequence diagrams and communication diagrams \\
This system can be really tricky when it comes to internal communication. There are a lot of services and it can be very hard to figure out what is happening; what services are used by whom, what is being communicated and how they do that. A sequence diagram can provide this information in very high detail and shows what happens if your system is invoked by for instance a user request or an external message coming in. The latter is the most interesting in our system. A sequence diagram showing what happens when a AIS sentence comes in would have been very usefull in understanding this system.

What about communcation diagrams? In addition to sequence diagrams, communication diagrams would also have been very usefull in understanding how the parts of the system communicate and provide a very good overview of the services. This is something that could be derived from the sequence diagrams, but that can be very tricky. Also, these diagrams can be really helpfull when interpreting sequence diagrams. You could compare it to a map and a route description. The Sequence diagram is the route description, telling you how to travel. The map is the overview of all the ways you could travel. Both of these artifacts will bring you to your destination, but the combination makes you fully understand.

\end{itemize}


\section*{Assignment 8}
While reviewing the system, we have come upon three major design faults. First, we'll name the problems and then go into further detail. The first fault we've come across was the fact that there are multiple `MessageX' classes. The second major design fault is the occurence of implementation details within the domain. The thrird problem we would like to discuss is the overcomplexity of the decoding pipeline.

\begin{itemize}
\item Multiple `message'-classes\\
The problem with these classes is that they are not in compliance with the 'Single Responsibility Principle'. This specifies that a class should have only one reason to change. Also the 'Liskov Subsitution Priciple'  is violated. The problem is that there are multiple classes built, while they have almost identical behaviour and properties. The proper way to implement these messages would be to use one superclass 'Message' with an attribute 'message\_type' which is an enum of possible messages. What another option could be is that we crate sub-classes of the superclass `Message'. The superclass holds the data and functions that are the same amongst all messages, while the sub-classes hold induvidual properties and functions. 
In this particular case, a good design would be to split message types into 2 types: Status update messages and location update messsages and make classes for them accordingly. Very important is that a class should only change, when the actual `thing' it models changes, and no other class is affected, but keeping in mind that two different classes that are the same, should be the same class.

\item MessageFactory inside domain\\
After mapping the design of this system it becomes clear that not everything is in the right place. Something is really wrong with the domain part of the system and that is the presence of the class `MessageFactory' within the domain module.

Why is this bad design? 
This is bad because a mix between domain concepts and implementation details has been created. For someone who is new to the system this can be very confusing. The proper way of dealing with this is called `Separation of concerns'. This principle states that the domain model should always be engineerd from the actual domain, and not from the system design. What this should accomplish is that the domain model describes the domain concepts, and the system model realizes the system concepts. In general, one must be carefull not to mix implementation details with domain concepts and visa versa.

\item Decoding pipeline overcomplexity\\
When looking at the decoding pipeline, we can see that it is pretty long. There are probably two reasons why this is done: First of all, it makes the program more modular and easier to maintain. Secondly, it would allow the maintainer to split the services over multiple machines to spread the workload.

Why is this a bad design?
When designing a system, one must take in to account what tasks it need to accomplish. If this system is only going to be used by one user, does it pay off to have such a long decoding pipeline? In this case it's not. The overhead produced by using a high amount of services has a negative inpact on performance, but it has a positive impact on scaleability. When scaleability is not desired, why use so many services? It increases the complexity of the system, and therefore has a negative impact on maintainability.
\end{itemize}



\end{document}
